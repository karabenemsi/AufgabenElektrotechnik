\documentclass[11pt,a4paper,titlepage,parskip=half]{scrreprt}

\usepackage[utf8x]{inputenc}
\usepackage[ngerman]{babel}
\usepackage{ucs}
\usepackage{amsmath}
\usepackage{amsfonts}
\usepackage{amssymb}
\usepackage{xcolor}
\usepackage{gensymb}
\usepackage{graphicx}
\usepackage{mdwlist}
\usepackage{siunitx}
\usepackage{nccmath}
\usepackage{subcaption}
\sisetup{locale=DE}
\usepackage[european]{circuitikz}
\usetikzlibrary{calc}
\usepackage{setspace}
\usepackage{geometry}


% Seitenränder -----------------------------------------------------------------
\setlength{\topskip}{\ht\strutbox} % behebt Warnung von geometry
\geometry{a4paper,left=30mm,right=30mm,top=30mm,bottom=35mm}

\usepackage[
automark, % Kapitelangaben in Kopfzeile automatisch erstellen
headsepline, % Trennlinie unter Kopfzeile
ilines % Trennlinie linksbündig ausrichten
]{scrpage2}

% Kopf- und Fußzeilen ----------------------------------------------------------
\pagestyle{scrheadings}
% chapterpagestyle gibt es nicht in scrartcl
\renewcommand{\chapterpagestyle}{scrheadings}
\clearscrheadfoot

% Kopfzeile
\renewcommand{\headfont}{\normalfont} % Schriftform der Kopfzeile
\ihead{\textsc{Versuch 1}\\[0.5ex] \textit{\headmark}}
\chead{E-Technik Praktikum\\Technische Informatik}
\ohead{\includegraphics*[scale=0.25]{../include/logo.png}}
\setlength{\headheight}{15mm} % Höhe der Kopfzeile
%\setheadwidth[0pt]{textwithmarginpar} % Kopfzeile über den Text hinaus verbreitern (falls Logo den Text überdeckt)

% Fußzeile
\ifoot{\today}
\cfoot{}
\ofoot{\pagemark}

% Abschnittsüberschriften im selben Stil wie beim Inhaltsverzeichnis einrücken
\renewcommand*{\othersectionlevelsformat}[3]{
    \makebox[\headingSpace][l]{#3\autodot}
}


%\onehalfspacing % Zeilenabstand 1,5 Zeilen
\frenchspacing % erzeugt ein wenig mehr Platz hinter einem Punkt

% Schusterjungen und Hurenkinder vermeiden
\clubpenalty = 10000
\widowpenalty = 10000
\displaywidowpenalty = 10000

% Aufzählungen anpassen
\renewcommand{\labelenumi}{\arabic{enumi}.}
\renewcommand{\labelenumii}{\arabic{enumi}.\arabic{enumii}.}
\renewcommand{\labelenumiii}{\arabic{enumi}.\arabic{enumii}.\arabic{enumiii}}


\makeatletter
    \def\pgf@circ@myvoltmeter@path#1{\pgf@circ@bipole@path{myvoltmeter}{#1}}
    \tikzset{myvoltmeter/.style = {\circuitikzbasekey, /tikz/to
                                   path=\pgf@circ@myvoltmeter@path}}
    \pgfcircdeclarebipole{}{\ctikzvalof{bipoles/voltmeter/height}}{myvoltmeter}{\ctikzvalof{bipoles/voltmeter/height}}{\ctikzvalof{bipoles/voltmeter/width}}{
        \def\pgf@circ@temp{right}
        \ifx\tikz@res@label@pos\pgf@circ@temp
            \pgf@circ@res@step=-1.2\pgf@circ@res@up
        \else
            \def\pgf@circ@temp{below}
            \ifx\tikz@res@label@pos\pgf@circ@temp
                \pgf@circ@res@step=-1.2\pgf@circ@res@up
            \else
                \pgf@circ@res@step=1.2\pgf@circ@res@up
            \fi
        \fi

        \pgfpathmoveto{\pgfpoint{\pgf@circ@res@left}{\pgf@circ@res@zero}}
        \pgfpointorigin \pgf@circ@res@other =  \pgf@x  \advance \pgf@circ@res@other by -\pgf@circ@res@up
        \pgfpathlineto{\pgfpoint{\pgf@circ@res@other}{\pgf@circ@res@zero}}
        \pgfusepath{draw}

        \pgfsetlinewidth{\pgfkeysvalueof{/tikz/circuitikz/bipoles/thickness}\pgfstartlinewidth}

            \pgfscope
                \pgfpathcircle{\pgfpointorigin}{\pgf@circ@res@up}
                \pgfusepath{draw}
            \endpgfscope

        \pgfsetlinewidth{\pgfstartlinewidth}
        \pgftransformrotate{90}
        \pgfsetarrowsend{latex}
        \pgfpathmoveto{\pgfpoint{\pgf@circ@res@other}{\pgf@circ@res@down}}
        \pgfpathlineto{\pgfpoint{-1.06\pgf@circ@res@other}{1.06\pgf@circ@res@up}}
        \pgfsetarrowsend{}


        \pgfpathmoveto{\pgfpoint{-\pgf@circ@res@other}{\pgf@circ@res@zero}}
        \pgfpathlineto{\pgfpoint{\pgf@circ@res@right}{\pgf@circ@res@zero}}

        \pgfnode{circle}{center}{\textbf{V}}{}{}
    }

    \def\pgf@circ@myammeter@path#1{\pgf@circ@bipole@path{myammeter}{#1}}
\tikzset{myammeter/.style = {\circuitikzbasekey, /tikz/to
        path=\pgf@circ@myammeter@path}}
\pgfcircdeclarebipole{}{\ctikzvalof{bipoles/voltmeter/height}}{myammeter}{\ctikzvalof{bipoles/voltmeter/height}}{\ctikzvalof{bipoles/voltmeter/width}}{
    \def\pgf@circ@temp{right}
    \ifx\tikz@res@label@pos\pgf@circ@temp
    \pgf@circ@res@step=-1.2\pgf@circ@res@up
    \else
    \def\pgf@circ@temp{below}
    \ifx\tikz@res@label@pos\pgf@circ@temp
    \pgf@circ@res@step=-1.2\pgf@circ@res@up
    \else
    \pgf@circ@res@step=1.2\pgf@circ@res@up
    \fi
    \fi
    
    \pgfpathmoveto{\pgfpoint{\pgf@circ@res@left}{\pgf@circ@res@zero}}
    \pgfpointorigin \pgf@circ@res@other =  \pgf@x  \advance \pgf@circ@res@other by -\pgf@circ@res@up
    \pgfpathlineto{\pgfpoint{\pgf@circ@res@other}{\pgf@circ@res@zero}}
    \pgfusepath{draw}
    
    \pgfsetlinewidth{\pgfkeysvalueof{/tikz/circuitikz/bipoles/thickness}\pgfstartlinewidth}
    
    \pgfscope
    \pgfpathcircle{\pgfpointorigin}{\pgf@circ@res@up}
    \pgfusepath{draw}
    \endpgfscope
    
    \pgfsetlinewidth{\pgfstartlinewidth}
    \pgftransformrotate{90}
    \pgfsetarrowsend{latex}
    \pgfpathmoveto{\pgfpoint{\pgf@circ@res@other}{\pgf@circ@res@down}}
    \pgfpathlineto{\pgfpoint{-1.06\pgf@circ@res@other}{1.06\pgf@circ@res@up}}
    \pgfsetarrowsend{}
    
    
    \pgfpathmoveto{\pgfpoint{-\pgf@circ@res@other}{\pgf@circ@res@zero}}
    \pgfpathlineto{\pgfpoint{\pgf@circ@res@right}{\pgf@circ@res@zero}}
    
    \pgfnode{circle}{center}{\textbf{A}}{}{}
}
\makeatother

\newcommand{\mathdirectcurrent}{\mathrel{\mathpalette\mathdirectcurrentinner\relax}}
\newcommand{\mathdirectcurrentinner}[2]{%
    \settowidth{\dimen0}{$#1=$}%
    \vbox to .85ex {\offinterlineskip
        \hbox to \dimen0{\hss\leaders\hrule\hskip.85\dimen0\hss}
        \vskip.35ex
        \hbox to \dimen0{\hss
            \leaders\hrule\hskip.17\dimen0
            \hskip.17\dimen0
            \leaders\hrule\hskip.17\dimen0
            \hskip.17\dimen0
            \leaders\hrule\hskip.17\dimen0
            \hss}
        \vfill
    }%
}
\newcommand{\textdirectcurrent}{\mathdirectcurrentinner{\textstyle}{}}

\newcommand{\spannung}[1]{\textcolor{blue}{#1}}
\newcommand{\strom}[1]{\textcolor{red}{#1}}
\newcommand{\widerstand}[1]{\textcolor{violet}{#1}}
\newcommand{\capacity}[1]{\textcolor{green}{#1}}
\newcommand{\glaettung}[1]{\textcolor{purple}{#1}}
\newcommand{\TODO}[1]{\textbf{\huge\textcolor{red}{!!!! #1 !!!!}}}

\newcommand{\vnumber}{1}
\newcommand{\vname}{Stromversorgungsschaltung}

\begin{document}
	    \begin{titlepage}
        \centering
        {\scshape\LARGE Hochschule Albstadt-Sigmaringen \par}
        {\scshape\large Studiengang Technische Informatik \par}
        \vspace{3cm}
        {\LARGE\bfseries Praktikum Elektrotechnik\par}
        \vspace{2cm}
        {\Huge\bfseries Versuch \vnumber\par}
        \vspace{1cm}
        {\Large \vname\par}
        \vspace{2cm}
        %\includegraphics[width=\textwidth]{example-image-1x1}\par
        \vfill

        % Bottom of the page
        {\large \today\par}
    \end{titlepage}


   \tableofcontents
	   
	\chapter{Einweggleichrichtung}
		\section{Einweggleichrichtung mit ohmscher Belastung ohne Kondensator}
		\paragraph{Messaufbau:}
		\begin{itemize*}
			\item 1 Widerstand $\widerstand{R} = 1\,k\Omega$
			\item 1 Widerstand $\widerstand{R_m} = 10\,\Omega$
			\item 1 Diode V1, Typ 1N4001
		\end{itemize*}		
		\begin{center}
			\begin{circuitikz}[scale=1.3]
				 \draw
				 (0,0) node[transformer,yscale=2] (T) {}
				 (T.B1) node[anchor=west] {}
				 (T.B2) node[anchor=west] {}
				 (T.A1) node[anchor=east] {230\,V 50\,Hz};
				 \ctikzset{label/align = smart}
				 \draw
				 (T.B1) to[short] (1,0)
				 (1,0) to[myvoltmeter, l^=$\spannung{U_{1}} {=} 16\,V$, *-*] (1,-3 |-T.B2)
				 (T.B1) to[D, l=1N 4001, i>^=$\strom{I_1}$] (4,0)
				 to[myvoltmeter, l^=$\spannung{U_2}$, *-*] (4,-3 |-T.B2)
				 node[rground]{}
				 to[R, l=$\widerstand{R_m}$] (T.B2)
				 (4,0) to[short] (6,0)
				 (6,0) to[R, l=$\widerstand{R}$, i>^=$\strom{I_2}$] (6,-3 |-T.B2)
				 (6,-3 |-T.B2) to[short] (4,-3 |-T.B2)
				 
				 % Oszillo Anschlüsse
				 (4,0) to[short, l_=Kanal Y2] (4,1)
				 (T.B1) to[short, l=Kanal Y1] (1,1-| T.B1)
				 (T.B2) to[short, l_=Osz. Gnd.] (1,-4 -| T.B2);
			\end{circuitikz}
		\end{center}
    
    
        \begin{table}[!hbtp]
            \caption{Oszillographeneinstellung}
            \label{tbl:oszillographeneinstellung}
            \renewcommand{\arraystretch}{1.3}
            \begin{tabular}{l|l}
                Kanal Y1 & $\SI{5}{\volt}$ pro Teil, Signal bei $\spannung{u_1(t)}$\\
                Kanal Y1 & $\SI{5}{\volt}$ pro Teil, Signal bei $\spannung{u_2(t)}$\\
                Kopplung & DC\\
                Trigger & Y1, in, norm, level\\
                Darstellung & Chopped\\
                Zeitablenkung & \SI{5}{\micro\second}
            \end{tabular}
        \end{table}
		
		\subsection{Messaufgaben}
			\subsubsection{Messaufgabe 1}
				\paragraph{Aufgabe:} Skizzieren Sie die Spannungs- und Stromverläufe \spannung{$U_1(t)$}, \spannung{$U_2(t)$} und \strom{$I_1(t)$}.
				\paragraph{Durchführung:} Schaltung aufbauen. $\spannung{U_1} = \SI{16}{\volt}$ einstellen mit Regler am roten Netztrafo. Oszillograph anschließen. Messen Sie den Diodenstrom \strom{$I_D(t)$} indirekt am Messwiderstand \widerstand{$R_m$}.

				
						
			\subsubsection{Messaufgabe 2}
                \paragraph{Aufgabe:} Zeichen Sie die Spannungsverläufe auf
            \subsubsection{Messaufgabe 3}
				\paragraph{Aufgabe:} Messen Sie mit dem Oszillograph und Multimeter
					\begin{table}[H]
						\caption{Messergebnisse Einweggleichrichtung ohne Kondensator}
						\label{tbl:messergebnisse1.1}
						\renewcommand{\arraystretch}{1.6}
                        \begin{center}
						\begin{tabular}{ll|l}
							\multicolumn{2}{l}{\textbf{Messgröße}} & \textbf{Messergebnis}\\ 
                            \multicolumn{3}{l}{Oszillograph:}\\\hline
							Frequenz der Eingangsspannung & $f$ & \\\hline
							Brummspannungsfrequenz & $f_{br}$ &\\\hline
							Scheitelwerte & \spannung{$U_{1_{max}}$} &\\\hline
							Scheitelwert & \spannung{$U_{2_{max}}$} &\\\hline
							Stromflusswinkel &  $\alpha [ \degree]$& \\\hline
							Brummspannung &  \spannung{$U_{brmax}$} &\\\hline
                                                        \multicolumn{3}{l}{Multimeter:}\\\hline
							Effektivwert & \spannung{$U_{1}$} &\\\hline
							Gleichspannung & \spannung{$U_{2-}$} & \\
						\end{tabular}
                    \end{center}
					\end{table}
		\subsection{Auswertung}
			\paragraph{Aufgabe 1:} Berechnen Sie aus den Messwerten das Verhältnis $\frac{\spannung{U_1}}{\spannung{U_{2-}}}$. Geben Sie den gemessenen und den theoretischen Wert an (mit Herleitung).
            
            \paragraph{Aufgabe 2:} Erklären Sie die indirekte Strommessung mit dem Oszillograph, und geben Sie den gemessenen und errechneten Wert an. Begründen Sie den Unterschied zwischen den Werten.
	
		\section{Einweggleichrichtung mit Glättungskondensator}
			
			\paragraph{Messaufbau}
			\begin{itemize*}
				\item 1 Widerstand $\widerstand{R} = 1\,k\Omega$
				\item 1 Widerstand $\widerstand{R_m} = 10\,\Omega$
				\item 1 Diode V1, Typ 1N4001
				\item 1 Kondensator $\capacity{C} = 100\,\mu F,40\,V Elektrolyt$
			\end{itemize*}
			\begin{center}
				\begin{circuitikz}[scale=1.3]
					\draw
					(0,0) node[transformer,yscale=2] (T) {}
					(T.B1) node[anchor=west] {}
					(T.B2) node[anchor=west] {}
					(T.A1) node[anchor=east] {230\,V 50\,Hz}
					;
					\ctikzset{label/align = smart}
					\draw
					(T.B1) to[short] (1,0)
					(1,0) to[myvoltmeter, l^=$\spannung{U_1}$] (1,-3 |-T.B2)
					(T.B1) to[D, l=1N4001, i>^=$\strom{I_1}$] (4,0)
					to[eC, l^=\capacity{C} , i>^=$\strom{I_C}$] (4,-3 |-T.B2)
					(3,0) to[short] (7,0)
					(5,0) to[R, l=$\widerstand{R}$, i>^=$\strom{I_2}$] (5,-3 |-T.B2)
					(5,0) to[short] (7,0)
					to[myvoltmeter, l^=$\spannung{U_2}$] (7,-3 |-T.B2)
					to[short] (4,-3 |-T.B2)
					to[R, l=$\widerstand{R_m}$] (T.B2)
					
					% Oszillo Anschlüsse
					(4,0) to[short, l_=Kanal Y2] (4,1)
					(T.B1) to[short, l=Kanal Y1] (1,1-| T.B1)
					(T.B2) to[short, l_=Osz. Gnd.] (1,-4 -| T.B2)
					;
				\end{circuitikz}
			\end{center}
			
			\subsection{Messaufgaben}
				\subsubsection{Messaufgabe 1}
					\paragraph{Aufgabe:} Messen Sie die Spannungs- und Stromverläufe  $\spannung{U_1(t)}, \spannung{U_2(t)}, \strom{I_2(t)} = \frac{\spannung{U_2(t)}}{\widerstand{R}}$ mit dem Oszillographen.
					\paragraph{Durchführung:} Schaltung aufbauen. $\spannung{U_1} = 16\,V$ einstellen.\\ 		

				\subsubsection{Messaufgabe 2}
					\paragraph{Aufgabe:} Messen sie mit dem Oszillograph und Multimeter
					\begin{table}[H]
						\caption{Messergebnisse Einweggleichrichtung mit Kondensator}
						\label{tbl:messergebnisse1.2}
						\renewcommand{\arraystretch}{1.6}
                        \begin{center}
						\begin{tabular}{ll|l}
							\multicolumn{2}{l}{\textbf{Messgröße}} & \textbf{Messergebnis}\\
                                                        \multicolumn{3}{l}{Oszillograph:}\\\hline
							Frequenz der Eingangsspannung & $f$ & \\\hline
							Brummspannungsfrequenz & $f_{br}$ &\\\hline
							Scheitelwerte & \spannung{$U_{1_{max}}$} &\\\hline
							Scheitelwert & \spannung{$U_{2_{max}}$} &\\\hline
							Stromflusswinkel &  $\alpha [ \degree]$&\\\hline
							Brummspannung &  \spannung{$U_{brmax}$} &\\\hline
                            \multicolumn{3}{l}{Multimeter:}\\\hline
							Effektivwert & \spannung{$U_{1}$} &\\\hline
							Gleichspannung & \spannung{$U_{2-}$} &\\
						\end{tabular}
                    \end{center}
					\end{table}
			\subsection{Auswertung}
				\paragraph{Aufgabe 1:} Bestätigen Sie die Näherung $\spannung{U_2} \approx  \sqrt{2} \cdot (\spannung{U_1} - 0,65) \cdot \cos(\frac{a}{2})$				
				
				\paragraph{Aufgabe 2:} Bestimmen Sie den Glättungsfaktor \glaettung{G}\\
					\begin{equation*}
						\glaettung{G} = 2 \cdot 3,14 \cdot f \cdot \capacity{C} \cdot \widerstand{R}
					\end{equation*}
					(siehe Anhang)
	
	\chapter{Brückengleichrichtung}
		\section{Brückengleichrichtung ohne Glättungskondensator}
			\paragraph{Messaufbau:}
			\begin{itemize*}
				\item 1 Widerstand $\widerstand{R} = 1\,k\Omega$
				\item 1 Widerstand $\widerstand{R_m} = 10\,\Omega$
				\item Brückengleichrichter Typ B80 C 1000/1500
			\end{itemize*}
			\begin{center}
				\begin{circuitikz}[scale=1.3]
					\draw
					(-0.5,4) node[transformer,yscale=1.2] (T) {}
					(T.B1) node[anchor=west] {}
					(T.B2) node[anchor=west] {}
					(T.A1) node[anchor=east] {}
					(0,4.2 -| T.base) node{\SI{230}{\volt}, \SI{50}{\hertz} $\rightarrow$ \SI{16}{\volt}, \SI{50}{\hertz}}
					;
					\ctikzset{label/align = smart}
					\draw
					(T.B1) to[open, v^=$\spannung{U_1}$] (T.B2)
					(1,3) to[Do] (2,4 |-T.B1)
					to[Do] (3,3)
					(1,3) to[Do] (2,2 |-T.B2)
					to[Do] (3,3)
					(T.B1) to[short, i_=$\strom{I_1}$] (2,4 |-T.B1)
					(T.B2) to[short] (2,2 |-T.B2)
					;
					\draw
					(3,3) to[short , i_=$\strom{I_2}$] (7,3)
					(5.5,3) to[R, l=$\widerstand{R}$, *-* ] (5.5,0)
					(7,3) to[myvoltmeter, l=$\spannung{U_2}$] (7,0)
					to[short] (4,0)
					to[R, l=$\widerstand{R_m}$] (1,0)
					to[short] (1,3)
					% Oszillo Anschlüsse
					(5.5,0) to[short, l_=Kanal Y2] (5.5,-1)
					(4,3) to[short, l_=Kanal Y1] (4,4)
					(1.5,0) to[short, l_=Osz. Gnd.] (1.5,-1)
					;
				\end{circuitikz}
			\end{center}
			
			\subsection{Messaufgaben}
			\subsubsection{Messaufgabe 1}
			\paragraph{Aufgabe:} Zeichnen Sie die Spannungs- und Stromveräufe $\spannung{U_1(t)}, \spannung{U_2(t)}$ und $\strom{I_2(t)}$ auf
			\paragraph{Durchführung:}  Schaltung aufbauen. $\spannung{U_1} = \SI{16}{\volt}$  einstellen. Oszillograph anschließen.\\
			
			
			\subsubsection{Messaufgabe 2}
			\paragraph{Aufgabe:} Messen sie mit dem Oszillograph und Multimeter
			\begin{table}[H]
				\caption{Messergebnisse Brückengleichrichtung ohne Glättungskondensator}
				\label{tbl:messergebnisse2.1}
				\renewcommand{\arraystretch}{1.6}
                \begin{center}
				\begin{tabular}{ll|l}
					\multicolumn{2}{l}{\textbf{Messgröße}} & \textbf{Messergebnis}\\
                                                \multicolumn{3}{l}{Oszillograph:}\\\hline
					Frequenz der Eingangsspannung & $f$ & \\\hline
					Brummspannungsfrequenz & $f_{br}$ & \\\hline
					Scheitelwerte & \spannung{$U_{1_{max}}$} & \\\hline
					Scheitelwert & \spannung{$U_{2_{max}}$} &\\\hline
					Stromflusswinkel &  $\alpha [ \degree]$& \\\hline
					Brummspannung &  \spannung{$U_{brmax}$} &\\\hline
                                                \multicolumn{3}{l}{Multimeter:}\\\hline
					Effektivwert & \spannung{$U_{1}$} &\\\hline
					Gleichspannung & \spannung{$U_{2-}$} & \\
				\end{tabular}
            \end{center}
			\end{table}
        
			\subsection{Auswertung}
			\paragraph{Aufgabe 1:} Berechnen Sie aus den Messwerten das Verhältnis $\frac{\spannung{U_1}}{\spannung{U_{2-}}}$. Geben Sie den theoretischen Wert an (Herleitung, Diodenspannung vernachlässigt).
            
		\section{Brückengleichrichtung mit Glättungskondensator}
			\paragraph{Messaufbau:}
			\begin{itemize*}
				\item 1 Widerstand $\widerstand{R} = 1\,k\Omega$
				\item 1 Widerstand $\widerstand{R_m} = 10\,\Omega$
				\item 1 Kondensator \capacity{C} = 33 µF, 40 V
				\item 1 Kondensator \capacity{C} = 100 µF, 40 V
				\item 1 Kondensator \capacity{C} = 220 µF, 40 V
				\item 1 Kondensator \capacity{C} = 1000 µF, 40 V
				\item Brückengleichrichter Typ B80 C 1000/1500
			\end{itemize*}
			\begin{center}
				\begin{circuitikz}[scale=1.3]
					\draw
					(-0.5,4) node[transformer,yscale=1.2] (T) {}
					(T.B1) node[anchor=west] {}
					(T.B2) node[anchor=west] {}
					(T.A1) node[anchor=east] {}
					(0,4.2 -| T.base) node{\SI{230}{\volt}, \SI{50}{\hertz} $\rightarrow$ \SI{16}{\volt}, \SI{50}{\hertz}}
					;
					\ctikzset{label/align = smart}
					\draw
					(T.B1) to[open, v^=$\spannung{U_1}$] (T.B2)
					(1,3) to[Do] (2,4 |-T.B1)
					to[Do] (3,3)
					(1,3) to[Do] (2,2 |-T.B2)
					to[Do] (3,3)
					(T.B1) to[short, i_=$\strom{I_1}$] (2,4 |-T.B1)
					(T.B2) to[short] (2,2 |-T.B2)
					;
					\draw
					(3,3) to[short] (4,3)
						  to[short, i_=$\strom{I_2}$] (5.5,3)
						  to[short] (7,3)
					(4,3) to[eC, l=$\capacity{C}$, *-* ] (4,0)
					(5.5,3) to[R, l=$\widerstand{R}$, *-* ] (5.5,0)
					(7,3) to[myvoltmeter, l=$\spannung{U_2}$] (7,0)
						  to[short] (4,0)
						  to[R, l=$\widerstand{R_m}$] (1,0)
						  to[short] (1,3)
					
					% Oszillo Anschlüsse
					(5.5,0) to[short, l_=Kanal Y2] (5.5,-1)
					(5,3) to[short, l=Kanal Y1] (5,4)
					(1.5,0) to[short, l_=Osz. Gnd.] (1.5,-1)
					;
				\end{circuitikz}
			\end{center}
			

			\subsection{Messaufgaben}
			\subsubsection{Messaufgabe 1}
			\paragraph{Aufgabe:}   Messen und skizzieren Sie für \capacity{C} mit \SI{33}{\mu\farad} die Spannungs- und Stromverläufe von $\spannung{U_2(t)}$ und $\strom{I_2(t)}$ auf.
			\paragraph{Durchführung:}  Schaltung aufbauen. $\spannung{U_1} = \SI{16}{\volt}$ einstellen. Werte messen und aufschreiben. Zeichnung anfertigen.

			\subsubsection{Messaufgabe 2}
			\paragraph{Aufgabe:}  Protokollieren Sie die Werte für verschiedene Größen des Kondensators $\capacity{C_1}$ in u.a. Tabelle. (Setzen Sie abwechseln die verschiedenen Kondensatoren in die Schaltung ein). 
            
    			\begin{table}[H]
    				\caption{Messwertetabelle Brückengleichrichtung mit Glättungskondensator}
    				\label{tbl:messergebnisse2.2}
    				\renewcommand{\arraystretch}{1.6}
                    \begin{center}
    				\begin{tabular}{l|c|c|c|c}
    					$\capacity{C_1}$ [\si{\mu\farad}] & \SI{33}{\mu\farad} &\SI{100}{\mu\farad} &\SI{220}{\mu\farad} &\SI{1000}{\mu\farad}\\ \hline
    					$f_{Eingang} [\si{\hertz}]$  &  &  &  & \\\hline
    					$f_{br} [\si{\hertz}]$ &  &  &  & \\\hline
    					$\spannung{U_{brss}} [\si{\volt}]$ &  &  &  &   \\\hline
    					$\mfrac{\spannung{U_1}}{\spannung{U_2}}$ &  &  &  & \\\hline
    					W ($10^{-2}$) & &  &  &  \\\hline
    					$\spannung{U_1} [\si{\volt}]$ &  &  &  & \\\hline
                        $\spannung{U_2} [\si{\volt}]$ &  &  &  & \\\hline
                        G &	 &  &  & 
    				\end{tabular}
                \end{center}
    			\end{table}
            
			\subsection{Auswertung}
			\paragraph{Aufgabe 1:}  Berechnen Sie die Verhältnisse $\mfrac{\spannung{U_1}}{\spannung{U_2}}$, $W =\mfrac{\spannung{U_{2w}}}{\spannung{U_2}}$, sowie den Glättungsfaktor G für obige Messreihe. Rechnen Sie mit $\spannung{U_{2w}} = \mfrac{\spannung{U_{2brss}}}{2,828}$. Beurteilen Sie die Ergebnisse in Bezug auf die Dimensionierung von Stromversorgungsschaltungen.\\
            
            
	\chapter{Siebschaltungen}
    Die Ausgangsspannungen und -Ströme der Gleichrichterschaltungen enthalten noch relativ große Wechselanteile (Brummanteile). Mit Hilfe von Siebschaltungen (Tiefpass Netzwerke) können diese Wechselanteile nochmals reduziert werden.
    	\section{RC-Siebung}
            \paragraph{Messaufbau:}
            \begin{itemize*}
                \item 1 Widerstand $\widerstand{R} = \SI{470}{\ohm}$
                \item 1 Widerstand $\widerstand{R_s} = $ ? $ \si{\ohm}$
                \item 1 Kondensator $\capacity{C_1} = \SI{22}{\mu\farad}, \SI{40}{\volt}$
                \item 1 Kondensator $\capacity{C_s} = $ ? $\si{\mu\farad}, \SI{40}{\volt}$
                \item Brückengleichrichter Typ B80 C 1000/1500
            \end{itemize*}
            \begin{center}
                \begin{circuitikz}[scale=1.3]
                    \draw
                    (-0.5,4) node[transformer,yscale=1.2] (T) {}
                    (T.B1) node[anchor=west] {}
                    (T.B2) node[anchor=west] {}
                    (T.A1) node[anchor=east] {}
                    (0,4.2 -| T.base) node{\SI{230}{\volt}, \SI{50}{\hertz} $\rightarrow$ \SI{16}{\volt}, \SI{50}{\hertz}}
                    ;
                    \ctikzset{label/align = smart}
                    \draw
                    (T.B1) to[open, v^=$\spannung{U_1}$] (T.B2)
                    (1,3) to[Do] (2,4 |-T.B1)
                    to[Do] (3,3)
                    (1,3) to[Do] (2,2 |-T.B2)
                    to[Do] (3,3)
                    (T.B1) to[short, i_=$\strom{I_1}$] (2,4 |-T.B1)
                    (T.B2) to[short] (2,2 |-T.B2)
                    ;
                    \draw
                    (3,3) to[short] (4,3)
                          to[R, l=$\widerstand{R_s}$, i_=$\strom{I_2}$] (6.2,3)
                          to[short] (8,3)
                    (3.5,3) to[eC, l_=$\capacity{C_1}$, *-* ] (3.5,0)
                    (6.2,3) to[eC, l_=$\capacity{C_s}$, *-* ] (6.2,0)
                    (8,3) to[myvoltmeter , l=$\spannung{U_3}$] (8,0)
                    to[short] (4,0)
                    to[R, l=$\widerstand{R_m}$] (1,0)
                    to[short] (1,3)
                    (4.5,3) to[myvoltmeter , l=$\spannung{U_2}$, *-*] (4.5,0)
                    (7,3) to[R , l=$\widerstand{R}$, *-*] (7,0)
                    ;
                \end{circuitikz}
            \end{center}
            
            \subsection{Messaufgaben}
            \subsubsection{Messaufgabe 1}
            \paragraph{Aufgabe:}  Für die Gleichrichterschaltung aus 2.2 ist ein RC-Siebglied auszulegen. Dimensionieren Sie den Serienwiderstand $\widerstand{R_s}$ (Widerstand, Leistung) und den Siebkondensator $\capacity{C_s}$ so, dass der Siebfaktor $s = \frac{\spannung{U_{2w}}}{\spannung{U_{3w}}}$ ca. $10$ beträgt. Rechnen Sie mit der im Anhang angegebenen Näherungsformel für RC-Siebung. Folgende Randbedingungen sind einzuhalten: der zusätzliche Spannungsabfall am Serienwiderstand Rs darf $10\,\%$ der Ausgangsspannung (bei Nennstrom) nicht überschreiten. maximale Ausgangslast $\widerstand{R} = \SI{470}{\ohm}$. Messen Sie die Verhältnisse bei einer Belastung von $\widerstand{R} = \SI{470}{\ohm}$ mit dem Oszillograph nach.
            \paragraph{Durchführung: } Schaltung aufbauen, Messwerte (Restwelligkeit) protokollieren und graphisch darstellen ($\spannung{U_1}, \spannung{U_2}, \spannung{U_3} $).
            
        
    \chapter{Spannungsstabilisierung}
        \section{Spannungsserienstabilisierung mit einem längsgeregeltem DC/DC-Wandler}
            \paragraph{Messaufbau:}
            \begin{itemize*}
                \item 1 Widerstand $\widerstand{R_{Last}} = \SI{56}{\ohm}, 10\,\%, \SI{3}{\watt}$
                \item 1 Widerstand $\widerstand{R_{Last}} = \SI{220}{\ohm}, 10\,\%, \SI{3}{\watt}$
                \item 1 Widerstand $\widerstand{R_{Last}} = \SI{470}{\ohm}, 10\,\%, \SI{3}{\watt}$
                \item 1 Widerstand $\widerstand{R_{Last}} = \SI{1,2}{\kilo\ohm}, 10\,\%, \SI{3}{\watt}$
                \item 1 Widerstand $\widerstand{R_1} = \SI{6,7}{\ohm}, 10\,\%$
                \item 1 Kondensator $\capacity{C_1} = \SI{100}{\mu\farad}, \SI{40}{\volt}$
                \item 1 Kondensator $\capacity{C_2} = \SI{22}{\mu\farad}, \SI{40}{\volt}$
                \item 1 Kondensator $\capacity{C_3} = \SI{0,47}{\mu\farad}, \SI{40}{\volt}$
                \item Brückengleichrichter Typ B80 C 1000/1500
                \item Spannungsregler IC1, 7805
            \end{itemize*}
            \begin{center}
                \begin{circuitikz}[scale=1.3]
                    %\draw
                    %(-0.5,4) node[transformer,yscale=1.2] (T) {}
                    (T.B1) node[anchor=west] {}
                    (T.B2) node[anchor=west] {}
                    (T.A1) node[anchor=east] {}
                    %(0,4.2 -| T.base) node{\SI{220}{\volt}, \SI{50}{\hertz} $\rightarrow$ \SI{16}{\volt}, \SI{50}{\hertz}}
                    %;
                    %\ctikzset{label/align = smart}
                    \draw
                    (T.B1) to[open, v^=$\spannung{U_1}$] (T.B2)
                    (1,3) to[Do] (2,4 |-T.B1)
                    to[Do] (3,3)
                    (1,3) to[Do] (2,2 |-T.B2)
                    to[Do] (3,3)
                    (T.B1) to[short, i_=$\strom{I_1}$] (2,4 |-T.B1)
                    (T.B2) to[short] (2,2 |-T.B2)
                    ;
                    \draw
                    
                    %Einmal außen herum
                   (3,3) to[short] (3.2,3)
                         to[R, l=$\widerstand{R_1}$] (4.5,3)
                         to[short] (5.2,3)
                         %Spannungsregler
                   (6.7,3) to[short] (7,3)
                           to[short] (8,3)
                           to[R, l=$\widerstand{R_{Last}}$] (8,0)
                           to[short] (3.2,0)
                           node[rground]{}
                           to[short, l=-] (1,0)
                           to[short] (1,3)
                           
                          
                   % Verbindungen in der Mitte
                   (3.2,3) to[eC, l_= $\capacity{C_1}$, *-*] (3.2,0)
                   (4.5,3) to[eC, l_= $\capacity{C_2}$, v^=$\spannung{U_2}$, *-*] (4.5,0)
                   (6,2.5) to[short, -*] (6,0)
                   (7.2, 3) to[eC, l_= $\capacity{C_3}$, *-*] (7.2,0)

                         
                         
                    %Spannungsregeler
                      (5.2,3.5)   rectangle (6.7,2.5)
                      (6,3) node [align=center] {In\qquad Out\\Gnd}
                      
                         
                    ;
                \end{circuitikz}
            \end{center}
            
            \subsection{Messaufgaben}
            \subsubsection{Messaufgabe 1}
            \paragraph{Aufgabe:}  Ausgangskennlinie $\spannung{U_3} = f(\widerstand{R_{Last}})$. Messen Sie mit dem Multimeter: $\spannung{U_{2-}}$ und $\spannung{U_{3-}}$. Beobachten Sie mit dem Oszillograph Ausgangsspannung $\spannung{U_{3-}}$.
            
            \paragraph{Durchführung:} Schaltung aufbauen. $\spannung{U_1} = \SI{16}{\volt}$ einstellen. Messwerte für die verschiedenen Widerstände in die Tabelle 4.1 eintragen.
            
            \begin{table}[H]
                \caption{Messwertetabelle Spannungsserienstabilisierung}
                \label{tbl:messergebnisse4.1}
                \renewcommand{\arraystretch}{1.6}
                \begin{center}
                    \begin{tabular}{l|c|c|c|c}
                        $\widerstand{R_{Last}} [\si{\ohm}]$ & 1200 & 470 & 220 & 56 \\ \hline
                        $\spannung{U_{2-}} [\si{\volt}]$  & \qquad\qquad\qquad  & \qquad\qquad\qquad & \qquad\qquad\qquad &\qquad\qquad\qquad\\\hline
                        $\spannung{U_{3-}} [\si{\volt}]$ &  &  &  & \\\hline
                        $\spannung{U_{3brss}} [\si{\milli\volt}]$ &  &  &  &  \\\hline
                        $P_v [\si{\watt}]$ &   &   &   &  \\\hline
                        Wirkungsgrad in \% &  &  &   &  \\
                    \end{tabular}
                \end{center}
            \end{table}
            
            \subsubsection{Messaufgabe 2}
            \paragraph{Aufgabe:} Spannungsregler - Wirkungsgrad. Lastwiderstand $\widerstand{R_{Last}} = \SI{100}{\ohm}$ Messen Sie mit dem  Multimeter: $\spannung{U_{2-}}$ und $\spannung{U_{3-}}$, Werte notieren.
            
            \subsubsection{Messaufgabe 3}
            \paragraph{Aufgabe:}  Ermitteln Sie die Eingangsspannung $\spannung{U_1}$ bei der die Schaltung für $\widerstand{R_{Last}} = \SI{56}{\ohm}$ noch einwandfrei regelt und geben Sie den Spannungswert an. Beobachten Sie dazu die Ausgangsspannung $\spannung{U_3(t)}$ mit dem Oszillograph. Stellen Sie zum Messen von $\spannung{U_{3brss}}$ den Oszillograph auf AC-Kopplung, um den Gleichspannungsanteil zu unterdrücken.
            
            \subsection{Auswertung}
        	\paragraph{Aufgabe 1: }  Berechnen Sie zu allen Messwerten die Verlustleistung $P_v = P_{ce}$ und den Wirkungsgrad des Spannungsreglers (Eigenverbrauch vernachlässigt). Tragen Sie die Daten in die Tabelle 4.1 ein und geben Sie die Berechnungen nachvollziehbar in der Ausarbeitung an!
            
           \listoftables
           
           
           \chapter{Anhang}
           \textbf{Gehört nicht zur Ausarbeitung, deshalb nicht übernehmen!}
           
           \section{Glättungsfaktor $\glaettung{G}$}
           Der Glättungsfaktor  $\glaettung{G}$ beschreibt bei Brummspannungen $\spannung{U_w}$ das Verhältnis des Eingangs zum Ausgang. Die Spannung am Verbraucher verläuft umso „glatter“ je größer die Zeitkonstante tc = CR im Verhältnis zur Periodendauer T der Eingangswechselspannung u (t) ist.
           
           Definition:
           \begin{align*}
           \glaettung{G} &= 2 \cdot \pi \cdot f \cdot \capacity{C} \cdot \widerstand{R} \quad \text{ mit }\\
           \widerstand{R} &= \text{Lastwiderstand} \\
           \capacity{C} &= \text{Glättungskondensator}\\
           f &= \text{Frequenz der Eingangswechselspannung}
           \end{align*}
           \section{Welligkeit einer Mischspannung}
           Der  gleichgerichteten  Wechselspannung  ist  ein  nichtsinusförmiger  Wechselanteil  (Brummspannung)  mit  der  Schwingungsbreite  $\spannung{U_{brss}}$ überlagert.  Der  Effektivwert  dieser  Spannung  kann  mittels  der  Fourieranalyse  (franz.  Mathematiker)  berechnet  werden.  Die  Brummspannung  enthält neben  der  Grundschwingung  $f_{br}$ Schwingungsanteile  mit  geradzahligem Vielfachen  der  Grundschwingungsfrequenz: $2f_{br}; 4f_{br}; 6f_{br}$.

           Definition:
           \begin{align*}
           W &= \frac{\spannung{U_w}}{\spannung{U_{\_}}}\\
           \spannung{U_2} &= \text{effektive Welligkeitsspannung}\\
           \spannung{U_{\_}} &= \text{arithmetischer Mittelwert der Mischspannung} 
           \end{align*}
           
           
           Welligkeit:\\
           Einweggleichrichtung (ohne C)   W = 1,21\\
           Brückengleichrichter  (ohne C)    W = 0,485\\
           
           \textbf{Hinweis:} Näherung  zur messtechnischen Bestimmung  der  Welligkeitsspannung $\spannung{U_w}$ aus  der  Brummspannung $\spannung{U_{brss}}$ bei der Ausnahme, dass $\spannung{U_2}$ enthält nur den Grundschwingungsanteil $f_{br}$ und $\spannung{U_{brss}} << \spannung{U_{\_}}$ dann:
           
           \begin{equation*}
           \spannung{U_w} = \frac{\spannung{U_{brss}}}{ 2 \cdot 1,4141}
           \end{equation*}
 
           \section{Stromflusswinkel "`$a$"'}
           Der  Gleichspannungsanteil  $U_{\_}$  am  Ausgang  einer  Gleichrichterschaltung  mit  Glättungskondensator hängt vom Stromflusswinkel "`$a$"', dh. von der Stromführungszeit durch die Gleichrichterdiode (Ladezeit des Kondensators) ab. Es gilt z.B für die Einwegschaltung:
           
           \begin{align*}
           \spannung{U_{2\_}} &= 1,4141 \cdot \spannung{U_1} \cdot \cos(\frac{a}{2})\\
           \spannung{U_1}&= \text{Effektivwert der sinusförmigen Eingangsspannung}\\
           \spannung{U_2}&= \text{arithmetischer Mittelwert der Ausgangsspannung}\\
           \end{align*}
           
           Beispiel: Gleichrichterschaltung mit Glättungskondensator und ohne Lastwiderstand.
           \begin{equation*}
           a = 0^0 \implies \spannung{U_{2\_}} = 1,4141 \cdot \spannung{U_1} = \spannung{U_{1\max}}
           \end{equation*}
           \section{Siebfaktor $S$}
           Siebglieder sind Tiefpassglieder. Der Siebfaktor gibt an, wievielmal größer die Welligkeitsspannung am Eingang $\spannung{U_{w1}}$ der Siebschaltung ist als am Ausgang $\spannung{U_{w2}}$.
           \begin{equation*}
           S = \frac{\spannung{U_{1w}}}{\spannung{U_{2w}}}
           \end{equation*}
           Eine Reihenschaltung aus einem Serienwiderstand $\widerstand{R_s}$ und einem parallelen Siebkondensator $\capacity{C_s}$:
           \begin{align*}
            S &= \frac{\spannung{U_{w1}}}{\spannung{U_{w2}}} = \sqrt{\frac{\widerstand{R_s}^2 + X_c^2}{X_c}}
           \end{align*}
           Näherung für $\widerstand{R_s} >> X_c$
           \begin{align*}
           S = \frac{\widerstand{R_s}}{X_c} = 2 \cdot \pi \cdot f_g \cdot \capacity{C_s} \cdot \widerstand{R_s}\\
           f_g = \text{Grundschwingung der Brummspannung}
           \end{align*}
           
\end{document}