\documentclass[11pt,a4paper,titlepage,parskip=half]{scrreprt}

\usepackage[utf8x]{inputenc}
\usepackage[ngerman]{babel}
\usepackage{ucs}
\usepackage{amsmath}
\usepackage{amsfonts}
\usepackage{amssymb}
\usepackage{xcolor}
\usepackage{gensymb}
\usepackage{graphicx}
\usepackage{mdwlist}
\usepackage{siunitx}
\usepackage{nccmath}
\usepackage{subcaption}
\sisetup{locale=DE}
\usepackage[european]{circuitikz}
\usetikzlibrary{calc}
\usepackage{setspace}
\usepackage{geometry}


% Seitenränder -----------------------------------------------------------------
\setlength{\topskip}{\ht\strutbox} % behebt Warnung von geometry
\geometry{a4paper,left=30mm,right=30mm,top=30mm,bottom=35mm}

\usepackage[
automark, % Kapitelangaben in Kopfzeile automatisch erstellen
headsepline, % Trennlinie unter Kopfzeile
ilines % Trennlinie linksbündig ausrichten
]{scrpage2}

% Kopf- und Fußzeilen ----------------------------------------------------------
\pagestyle{scrheadings}
% chapterpagestyle gibt es nicht in scrartcl
\renewcommand{\chapterpagestyle}{scrheadings}
\clearscrheadfoot

% Kopfzeile
\renewcommand{\headfont}{\normalfont} % Schriftform der Kopfzeile
\ihead{\textsc{Versuch 1}\\[0.5ex] \textit{\headmark}}
\chead{E-Technik Praktikum\\Technische Informatik}
\ohead{\includegraphics*[scale=0.25]{../include/logo.png}}
\setlength{\headheight}{15mm} % Höhe der Kopfzeile
%\setheadwidth[0pt]{textwithmarginpar} % Kopfzeile über den Text hinaus verbreitern (falls Logo den Text überdeckt)

% Fußzeile
\ifoot{\today}
\cfoot{}
\ofoot{\pagemark}

% Abschnittsüberschriften im selben Stil wie beim Inhaltsverzeichnis einrücken
\renewcommand*{\othersectionlevelsformat}[3]{
    \makebox[\headingSpace][l]{#3\autodot}
}


%\onehalfspacing % Zeilenabstand 1,5 Zeilen
\frenchspacing % erzeugt ein wenig mehr Platz hinter einem Punkt

% Schusterjungen und Hurenkinder vermeiden
\clubpenalty = 10000
\widowpenalty = 10000
\displaywidowpenalty = 10000

% Aufzählungen anpassen
\renewcommand{\labelenumi}{\arabic{enumi}.}
\renewcommand{\labelenumii}{\arabic{enumi}.\arabic{enumii}.}
\renewcommand{\labelenumiii}{\arabic{enumi}.\arabic{enumii}.\arabic{enumiii}}


\makeatletter
    \def\pgf@circ@myvoltmeter@path#1{\pgf@circ@bipole@path{myvoltmeter}{#1}}
    \tikzset{myvoltmeter/.style = {\circuitikzbasekey, /tikz/to
                                   path=\pgf@circ@myvoltmeter@path}}
    \pgfcircdeclarebipole{}{\ctikzvalof{bipoles/voltmeter/height}}{myvoltmeter}{\ctikzvalof{bipoles/voltmeter/height}}{\ctikzvalof{bipoles/voltmeter/width}}{
        \def\pgf@circ@temp{right}
        \ifx\tikz@res@label@pos\pgf@circ@temp
            \pgf@circ@res@step=-1.2\pgf@circ@res@up
        \else
            \def\pgf@circ@temp{below}
            \ifx\tikz@res@label@pos\pgf@circ@temp
                \pgf@circ@res@step=-1.2\pgf@circ@res@up
            \else
                \pgf@circ@res@step=1.2\pgf@circ@res@up
            \fi
        \fi

        \pgfpathmoveto{\pgfpoint{\pgf@circ@res@left}{\pgf@circ@res@zero}}
        \pgfpointorigin \pgf@circ@res@other =  \pgf@x  \advance \pgf@circ@res@other by -\pgf@circ@res@up
        \pgfpathlineto{\pgfpoint{\pgf@circ@res@other}{\pgf@circ@res@zero}}
        \pgfusepath{draw}

        \pgfsetlinewidth{\pgfkeysvalueof{/tikz/circuitikz/bipoles/thickness}\pgfstartlinewidth}

            \pgfscope
                \pgfpathcircle{\pgfpointorigin}{\pgf@circ@res@up}
                \pgfusepath{draw}
            \endpgfscope

        \pgfsetlinewidth{\pgfstartlinewidth}
        \pgftransformrotate{90}
        \pgfsetarrowsend{latex}
        \pgfpathmoveto{\pgfpoint{\pgf@circ@res@other}{\pgf@circ@res@down}}
        \pgfpathlineto{\pgfpoint{-1.06\pgf@circ@res@other}{1.06\pgf@circ@res@up}}
        \pgfsetarrowsend{}


        \pgfpathmoveto{\pgfpoint{-\pgf@circ@res@other}{\pgf@circ@res@zero}}
        \pgfpathlineto{\pgfpoint{\pgf@circ@res@right}{\pgf@circ@res@zero}}

        \pgfnode{circle}{center}{\textbf{V}}{}{}
    }

    \def\pgf@circ@myammeter@path#1{\pgf@circ@bipole@path{myammeter}{#1}}
\tikzset{myammeter/.style = {\circuitikzbasekey, /tikz/to
        path=\pgf@circ@myammeter@path}}
\pgfcircdeclarebipole{}{\ctikzvalof{bipoles/voltmeter/height}}{myammeter}{\ctikzvalof{bipoles/voltmeter/height}}{\ctikzvalof{bipoles/voltmeter/width}}{
    \def\pgf@circ@temp{right}
    \ifx\tikz@res@label@pos\pgf@circ@temp
    \pgf@circ@res@step=-1.2\pgf@circ@res@up
    \else
    \def\pgf@circ@temp{below}
    \ifx\tikz@res@label@pos\pgf@circ@temp
    \pgf@circ@res@step=-1.2\pgf@circ@res@up
    \else
    \pgf@circ@res@step=1.2\pgf@circ@res@up
    \fi
    \fi
    
    \pgfpathmoveto{\pgfpoint{\pgf@circ@res@left}{\pgf@circ@res@zero}}
    \pgfpointorigin \pgf@circ@res@other =  \pgf@x  \advance \pgf@circ@res@other by -\pgf@circ@res@up
    \pgfpathlineto{\pgfpoint{\pgf@circ@res@other}{\pgf@circ@res@zero}}
    \pgfusepath{draw}
    
    \pgfsetlinewidth{\pgfkeysvalueof{/tikz/circuitikz/bipoles/thickness}\pgfstartlinewidth}
    
    \pgfscope
    \pgfpathcircle{\pgfpointorigin}{\pgf@circ@res@up}
    \pgfusepath{draw}
    \endpgfscope
    
    \pgfsetlinewidth{\pgfstartlinewidth}
    \pgftransformrotate{90}
    \pgfsetarrowsend{latex}
    \pgfpathmoveto{\pgfpoint{\pgf@circ@res@other}{\pgf@circ@res@down}}
    \pgfpathlineto{\pgfpoint{-1.06\pgf@circ@res@other}{1.06\pgf@circ@res@up}}
    \pgfsetarrowsend{}
    
    
    \pgfpathmoveto{\pgfpoint{-\pgf@circ@res@other}{\pgf@circ@res@zero}}
    \pgfpathlineto{\pgfpoint{\pgf@circ@res@right}{\pgf@circ@res@zero}}
    
    \pgfnode{circle}{center}{\textbf{A}}{}{}
}
\makeatother

\newcommand{\mathdirectcurrent}{\mathrel{\mathpalette\mathdirectcurrentinner\relax}}
\newcommand{\mathdirectcurrentinner}[2]{%
    \settowidth{\dimen0}{$#1=$}%
    \vbox to .85ex {\offinterlineskip
        \hbox to \dimen0{\hss\leaders\hrule\hskip.85\dimen0\hss}
        \vskip.35ex
        \hbox to \dimen0{\hss
            \leaders\hrule\hskip.17\dimen0
            \hskip.17\dimen0
            \leaders\hrule\hskip.17\dimen0
            \hskip.17\dimen0
            \leaders\hrule\hskip.17\dimen0
            \hss}
        \vfill
    }%
}
\newcommand{\textdirectcurrent}{\mathdirectcurrentinner{\textstyle}{}}

\newcommand{\spannung}[1]{\textcolor{blue}{#1}}
\newcommand{\strom}[1]{\textcolor{red}{#1}}
\newcommand{\widerstand}[1]{\textcolor{violet}{#1}}
\newcommand{\capacity}[1]{\textcolor{green}{#1}}
\newcommand{\glaettung}[1]{\textcolor{purple}{#1}}
\newcommand{\TODO}[1]{\textbf{\huge\textcolor{red}{!!!! #1 !!!!}}}

\newcommand{\vnumber}{2}
\newcommand{\vname}{Gleichstromnetzwerke und Bauelemente}

\begin{document}
	    \begin{titlepage}
        \centering
        {\scshape\LARGE Hochschule Albstadt-Sigmaringen \par}
        {\scshape\large Studiengang Technische Informatik \par}
        \vspace{3cm}
        {\LARGE\bfseries Praktikum Elektrotechnik\par}
        \vspace{2cm}
        {\Huge\bfseries Versuch \vnumber\par}
        \vspace{1cm}
        {\Large \vname\par}
        \vspace{2cm}
        %\includegraphics[width=\textwidth]{example-image-1x1}\par
        \vfill

        % Bottom of the page
        {\large \today\par}
    \end{titlepage}

   \tableofcontents

  \chapter{Der belastete Spannungsteiler}


    \section{Einfluss der Belastung eines Spannungsteilers auf die Linearität der Ausgangsspannung}

      \paragraph{Messaufbau:}
          \begin{itemize*}
            \item 1 Widerstand $\widerstand{R_A} = \SI{100}{\kilo\ohm}$
            \item 1 Widerstand $\widerstand{R_A} = \SI{1}{\kilo\ohm}$
            \item 1 Potentiometer $\widerstand{R_P} = \SI{100}{\kilo\ohm}$
            \item 4 Messgeräte
          \end{itemize*}
          \begin{center}
            \begin{circuitikz}[scale=1.3]
                \draw
                (0,3) to[short] (1,3)
                      to[ammeter, i=$\strom{I_E}$] (3,3)
                      to[potentiometer, l_=$\widerstand{R_P}$] (3,0)
                (1,3) to[myvoltmeter, l=$\spannung{U_V}$ ,*-*] (1,0)
                (0,3) to[open, v=$\spannung{U_V}$] (0,0)
                (3,0) to[short] (0,0)
                (3.2,1.5) to[short] (3.7,1.5)
                          to[short] (3.7,3)
                          to[short] (4.3,3)
                          to[ammeter, i=$\strom{I_A}$] (6.5,3)
                          to[R, l=$\widerstand{R_A}$] (6.5,0)
                          to[short] (3,0)
                (4.3,3) to[myvoltmeter, l=$\spannung{U_A}$ ,*-*] (4.3,0)
                ;
            \end{circuitikz}
          \end{center}

      \subsection{Messaufgaben}
        \subsubsection{Messaufgabe M1}
          \paragraph{Aufgabe:} Ausgangsspannung $\spannung{U_A}$, Ströme $\strom{I_E}$ und $\strom{I_A}$ bei verschiedenen Potentiometerstellungen
          messen.
          \paragraph{Durchführung:}Aufbau der Schaltung. Versorgungsspannung $\spannung{U_V} = \SI{12}{\volt}$ einstellen. Jeweils eine
          Messreihe für $\widerstand{R_A} = \infty$, $\SI{1}{\kilo\ohm}$ und $\SI{100}{\kilo\ohm}$ aufnehmen.
          \paragraph{Ergebnisse:}
            \begin{center}
              \begin{table}[H]
                \caption{Messwertetabelle zur Messaufgabe 2.1.M1}
                \label{tbl:messergebnisse1.1}
                \renewcommand{\arraystretch}{1.5}
                \begin{tabular}{c|c|c|c|c|c|c|c|c|c}
                  Potenziometerstellung & \multicolumn{3}{c|}{\textbf{$\widerstand{R_A} = \infty$}} & \multicolumn{3}{c|}{\textbf{$\widerstand{R_A}$ = \SI{100}{\kilo\ohm}}}& \multicolumn{3}{c}{\textbf{$\widerstand{R_A}$ = \SI{1}{\kilo\ohm}}}\\ \hline
                                & $\strom{I}$ & $\spannung{U_A}$ & $\strom{I_A}$ & $\strom{I}$ & $\spannung{U_A}$ & $\strom{I_A}$ & $\strom{I}$ & $\spannung{U_A}$ & $\strom{I_A}$\\\hline
                                A & \quad\quad\quad   & \quad\quad\quad & \quad\quad\quad & \quad\quad\quad &\quad\quad\quad  &\quad\quad\quad  & \quad\quad\quad & \quad\quad\quad &\quad\quad\quad \\\hline
                                B &  & &  &  &  &  &  & & \\\hline
                                C & & & & & & & & & \\\hline
                                D & & & & & & & & & \\\hline
                                E & &  &  & &  & & &  & \\\hline
                                F & & & & & & & & & \\\hline
                                G & & & & & & & & & \\
                \end{tabular}
              \end{table}
            \end{center}
        \subsection{Auswertung}
          \subsubsection{Aufgabe 1:} Stellen Sie die Ausgangsspannung $\spannung{U_A}$ in Abhängigkeit der Potentiometerstellung
          mit $\widerstand{R_A}$ als Parameter graphisch dar.
          \subsubsection{Aufgabe 2:}  Für eine elektrische Wegerfassung mit Potentiometer soll die Zuordnung von Potentiometerstellung und Ausgangsspannung möglichst linear sein. Dem Potentiometer ist eine Auswerteelektronik nachgeschaltet. Was ist bei der Auslegung des Eingangswiderstandes für diese Elektronik zu beachten?

    \chapter{Ersatzspannungsquelle, Leistungsanpassung.}


        \section{Ersatzspannungsquelle}

          \paragraph{Messaufbau:}
            \begin{itemize*}
                \item 1 Widerstand $\widerstand{R_A} = \SI{100}{\ohm}$
                \item 1 Widerstand $\widerstand{R_A} = \SI{4,7}{\kilo\ohm}$
                \item 1 Widerstand $\widerstand{R_A} = \SI{10}{\kilo\ohm}$
                \item 1 Potentiometer $\widerstand{R_P} = \SI{4,7}{\kilo\ohm}$
                \item 1 unbekanntes Netzwerk
            \end{itemize*}
            \begin{center}
                \begin{circuitikz}[scale=1.3]
                    \draw
                    (0,3) to[short, i=$\strom{I_E}$] (1,3)
                    (1,0) to[short] (0,0)
                    (3,3) to[short, i=$\strom{I_A}$] (5,3)
                          to[potentiometer, l_=$\widerstand{R_P}$, -*] (5,0)
                          to[short] (6,0)
                          to[short] (6,1.5)
                          to[short] (5.2,1.5)
                    (5,0) to[short] (3,0)
                    (3,3) to[open, v^=$\spannung{U_A}$] (3,0)
                    (0,3) to[open, v^=$\spannung{U_V}$] (0,0)
                    (1,3.2) rectangle (3,-0.2)
                    (2,1.5) node [align=center] {Unbekanntes\\ Netzwerk}
                    ;
                \end{circuitikz}
            \end{center}

          \subsection{Messaufgaben}
            \subsubsection{Messaufgabe M1}
              \paragraph{Aufgabe:} Beschreibung eines unbekannten Netzwerkes als Ersatzspannungsquelle. Dazu
              Kennlinie der Klemmenspannung $\spannung{U_A} = f(\strom{I_A})$ (mit Leerlaufspannung $\spannung{U_A} = 0$)
              aufnehmen.

              \paragraph{Durchführung:} Messschaltung aufbauen. Versorgungsspannung $\spannung{U_V} = \SI{12}{\volt}$ einstellen, notfalls
              während der Messung nachregeln. Laststrom '$\strom{I_A}$ mit dem Lastwiderstand $\widerstand{R_P}$ in
              ca. \SI{1}{\milli\ampere} Schritten ändern. Strom $\strom{I_E}$ für eine spätere Leistungsberechnung mitmessen.
              Messwerte protokollieren.

              \paragraph{Ergebnisse:}
                \begin{center}
                    \begin{table}[H]
                        \caption{Messwertetabelle zur Messaufgabe 2.1.M1}
                        \label{tbl:messergebnisse2.1}
                        \renewcommand{\arraystretch}{1.5}
                         \begin{center}
                            \begin{tabular}{c|c|c|c|c|c|c}
                                $\strom{I_A} [\si{\milli\ampere}]$  &
                                $\strom{I_E} [\si{\milli\ampere}]$ &
                                $\spannung{U_A} [\si{\volt}]$ &
                                $\widerstand{R_A} = \mfrac{\spannung{U_A}}{\strom{I_A}} [\si{\ohm}]$ &
                                $\widerstand{R_I} = \mfrac{\triangle\spannung{U_A}}{\triangle\strom{I_A}} [\si{\ohm}]$ &
                                Wirkungsgrad [\%] &% $=\mfrac{\spannung{U_A} \cdot \frac{\strom{I_A}}{1000}}{\SI{12}{\volt} \cdot \frac{\strom{I_E}}{1000}}$ &
                                P [\si{\milli\watt}] %= $\spannung{U_A} \cdot \strom{I_A}$
                                \\ \hline

                                \quad\quad\quad & \quad\quad\quad &\quad\quad\quad & \quad\quad\quad & \quad\quad\quad & \quad\quad\quad & \quad\quad\quad\\\hline
                                 &  &  &  &  &  & \\\hline
                                 &  &  &  &  &  & \\\hline
                                 &  &  &  &  &  & \\\hline
                                 &  &  &  &  &  & \\\hline
                                 &  &  &  &  &  & \\\hline
                                 &  &  &  &  &  & \\\hline
                                 &  &  &  &  &  & \\\hline
                                 &  &  &  &  &  & \\\hline
                                 &  &  &  &  &  & \\
                            \end{tabular}
                        \end{center}
                    \end{table}
                \end{center}
           \subsubsection{Messaufgabe M2:}
               \paragraph{Aufgabe:}Belasten Sie das Netzwerk mir $\widerstand{R_A} = \SI{10}{\kilo\ohm}, \SI{4,7}{\kilo\ohm}$ und $\SI{100}{\ohm}$.
               Protokollieren Sie die Messwerte.

               \paragraph{Ergebnisse:}
                   \begin{center}
                       \begin{table}[!hbtp]
                           \caption{Messwertetabelle zur Messaufgabe 2.1.M2}
                           \label{tbl:messergebnisse2.2}
                           \renewcommand{\arraystretch}{1.3}
                           \begin{center}
                               \begin{tabular}{c|c|c|c|c}
                                   $\widerstand{R_A}$&
                                   $\strom{I_A} [\si{\milli\ampere}]$  &
                                   $\strom{I_E} [\si{\milli\ampere}]$ &
                                   $\spannung{U_A} [\si{\volt}]$ &
                                   $\spannung{U_V} [\si{\volt}]$ \\ \hline

                                   \SI{10}{\kilo\ohm} & \qquad\qquad & \qquad\qquad & \qquad\qquad & \qquad\qquad\\\hline
                                   \SI{4,7}{\kilo\ohm} &  &  &  &\\\hline
                                   \SI{100}{\ohm} &  &&  & \\

                               \end{tabular}
                           \end{center}
                       \end{table}
                   \end{center}
         \subsection{Auswertung}
           \subsubsection{Aufgabe 1:} Stellen Sie die Kennlinie $\spannung{U_A} = f(\strom{I_A})$ aus Messaufgabe 1 graphisch dar.

           \subsubsection{Aufgabe 2:} Beschreiben Sie das Netzwerk als Ersatzspannungsquelle. (Ersatzschaltbild grafisch darstellen und mathematische Beschreibung für die Ausgangsspannung $\spannung{U_A}$ und den Innenwiderstand $\widerstand{R_I}$). Zum Beispiel:\\
          
            \begin{center}
                \begin{circuitikz}[scale=1.3]
                    \draw
                    (0,3) to[short, i=$\strom{I_E}$] (3,3)
                          to[R, l=$\widerstand{R_{I_1}}$, v=$\spannung{U_I}$] (3,1.5)
                    (3,0) to[short] (0,0)
                    (3,1.5) to[short, i=$\strom{I_A}$] (5,1.5)
                    to[potentiometer, l_=$\widerstand{R_P}$, -*] (5,0)
                    to[short] (6,0)
                    to[short] (6,0.75)
                    to[short] (5.2,0.75)
                    (5,0) to[short] (3,0)
                    (3,1.5) to[R,l_=$\widerstand{R_{I_2}}$] (3,0)
                    (0,3) to[open, v^=$\spannung{U_V}$] (0,0)
                    (3.75,1.5) to[open, v^=$\spannung{U_A}$] (3.75,0)

                    (1,3.8) rectangle (3.8,-0.5)
                    %(2,1.5) node [align=center] {Unbekanntes\\ Netzwerk}
                    ;
                \end{circuitikz}
            \end{center}
        
            \begin{align*}
               \widerstand{R_I} &= ?
            \end{align*}

            \begin{align*}
                \spannung{U_A} &= \spannung{U_V} \cdot x
            \end{align*}

           \subsubsection{Aufgabe 3:} Berechnen Sie aus den Messwerten für alle Lastfälle den Innenwiderstand $\widerstand{R_i}$, des als Ersatzspannungsquelle beschriebenen Netzwerkes.
           
           
           \subsubsection{Aufgabe 4:} Welche und wie viele Messpunkte sind zur messtechnischen Ermittlung der Ersatzspannungsquelle eines beliebigen linearen Netzwerkes notwendig? Bitte die Werte Angeben!
           
           \subsubsection{Aufgabe 5:} Berechnen Sie für das Netzwerk den Wirkungsgrad aus den Messwerten von Aufgabe 1. Tragen Sie diese Werte in die Messwertetabelle ein. Stellen Sie die Lastkurve $P_A$ und den Wirkungsgrad in Abhängigkeit von $\widerstand{R_A}$ und $\strom{I_A}$ graphisch dar.\\
           
           Wirkungsgrad $=\frac{\text{Nutzleistung}}{\text{Eingespeiste Leistung}} = \frac{P_a}{P_e}$
           


           \subsubsection{Aufgabe 6:} Ermitteln Sie aus der Lastkurve den Punkt für Leistungsanpassung (Übertragung der größten Nutzleistung, nachrichtentechnische Anpassung) Wie groß ist dabei der Wirkungsgrad? Wie groß ist das Verhältnis $\frac{\widerstand{R_I}}{\widerstand{R_A}}$?\\
           $\widerstand{R_I}$: Innenwiderstand des als Ersatzspannungsquelle beschriebenen Netzwerkes
           

           \subsubsection{Aufgabe 7:} Vergleichen Sie den messtechnisch gewonnenen Wert für die Leistungsanpassung mit dem theoretischen? Stellen Sie dazu eine Gleichung für die Nutzleistung $P_A$ in Abhängigkeit von $\widerstand{R_P}$ auf. Bestimmen Sie daraus das Maximum.


    \chapter{Nichtlineare Bauelemente}


        \section{Strom- Spannung- Kennlinie einer Glühbirne}
          \paragraph{Messaufbau:}
            \begin{itemize*}
                \item 1 Glühbirne (\SI{12}{\volt}, \SI{3}{\watt}) mit Fassung
            \end{itemize*}
            \begin{center}
                \begin{circuitikz}[scale=1.3]
                    \draw
                    (0,2) to[ammeter, i=$\strom{I}$] (2,2)
                          to[short] (4,2)
                          to[lamp] (4,0)
                          to[short] (0,0)
                     (2,2) to[myvoltmeter, l=$\spannung{U_V}$, *-*] (2,0)
                    ;
                \end{circuitikz}
            \end{center}

          \subsection{Messaufgaben}
            \subsubsection{Messaufgabe M1}
              \paragraph{Aufgabe:} Nehmen Sie die Strom-Spannungs-Kennlinie $\strom{I} = f(\spannung{U_V})$ einer Glühbirne auf.
              \paragraph{Durchführung:} Messschaltung aufbauen. Die Spannung \spannung{U} am Bauelement einstellen. Ändern Sie
              dazu die Versorgungsspannung $\spannung{U}$ in \SI{1}{\volt} Schritten von ca. \SI{0}{\volt} bis \SI{12}{\volt}.
              Messwertetabelle für $\spannung{U}$, $\strom{I}$ und $\widerstand{R_A} = \mfrac{\spannung{U}}{\strom{I}}$ anlegen.
              \paragraph{Ergebnisse:}
                   \begin{center}
                        \begin{table}[H]
                            \caption{Messwertetabelle zur Messaufgabe 3.1.M1}
                            \label{tbl:messergebnisse3.1}
                            \renewcommand{\arraystretch}{1.5}
                            \begin{center}
                                \begin{tabular}{c|c|c}
                                    $\spannung{U} [\si{\volt}]$  &
                                    $\strom{I} [\si{\milli\ampere}]$ &
                                    $\widerstand{R_A} = \mfrac{\spannung{U}}{\strom{I}} [\si{\ohm}]$\\ \hline

                                    \quad\quad\quad & \quad\quad\quad & \quad\quad\quad\\\hline
                                     &  & \\\hline
                                     &  & \\\hline
                                     &  & \\\hline
                                     & & \\\hline
                                     &  & \\\hline
                                     &  & \\\hline
                                     & & \\\hline
                                     &  & \\\hline
                                     &  & \\\hline
                                     &  & \\\hline
                                     &  & \\\hline
                                     &  & \\
                                \end{tabular}
                            \end{center}
                        \end{table}
                    \end{center}
         \subsection{Auswertung}
           \subsubsection{Aufgabe 1} Zeichnen Sie $\spannung{U} = f (\strom{I})$


           \subsubsection{Aufgabe 2} Führen Sie eine Näherung durch, in dem Sie den Strom/ Spannungsverlauf der Glühbirne durch 2 lineare Geradenstücke ersetzen. Beschreiben Sie damit das elektrische Verhalten der Glühbirne (Ermittlung der Widerstände für diese 2 Bereiche). Welche Widerstandsarten können Sie hieraus ableiten? Begründen Sie es.
          
        
        \section{Kennlinie einer Z-Diode (Z - Diode in Durchlassrichtung)}
          \paragraph{Messaufbau:}
            \begin{itemize*}
                \item 1 Widerstand $\widerstand{R} = \SI{470}{\ohm}, \SI{2}{\watt}$
                \item 1 Z-Diode, Typ ZPD6V2 oder ZPD5V2
            \end{itemize*}
            \begin{center}
                \begin{circuitikz}[scale=1.3]
                    \draw
                    (0,2) to[R, l=$\widerstand{R}$] (1,2)
                    to[ammeter, i=$\strom{I}$] (3,2)
                    to[short] (5,2)
                    (5,2) to[zDo, v=$\spannung{U_D}$] (5,0)
                    (5,0)to[short] (0,0)
                    (3,2) to[myvoltmeter, l=$\spannung{U_V}$, *-*] (3,0)
                    ;
                \end{circuitikz}
            \end{center}

          \subsection{Messaufgaben}
            \subsubsection{Messaufgabe M1}
              \paragraph{Aufgabe:} Nehmen Sie die Kennlinie der Z - Diode $\spannung{U_D} = f(\strom{I_D})$ in Durchlassrichtung auf
              \paragraph{Durchführung:} Messschaltung aufbauen. Die Versorgungsspannung $\spannung{U_V} = \SI{0}{\volt}$ einstellen. Mit $\spannung{U_V}$ die Diodenspannung $\spannung{U_D}$ schrittweise erhöhen ( im Durchlassbereich \SI{0,05}{\volt} Schritte) bis der Diodenstrom $\strom{I_D} = \SI{40}{\milli\ampere}$ beträgt. Messwertetabelle für $\spannung{U_D}$ , $\strom{I_D}$ und $\widerstand{R_D} = \mfrac{\spannung{U_D}}{\strom{I_D}}$ anlegen oder gegebene benutzen.
              \paragraph{Ergebnisse:}
                \begin{center}
                    \begin{table}[!hbtp]
                        \caption{Messwertetabelle zur Messaufgabe 3.2.M1}
                        \label{tbl:messergebnisse3.2}
                        \renewcommand{\arraystretch}{1.3}
                        \begin{center}

                        \begin{tabular}{c|c|c|c}
                            $\spannung{U_V} [\si{\volt}]$ &
                            $\spannung{U_D}[\si{\volt}]$ &
                            $\strom{I_D} [\si{\milli\ampere}]$ &
                            $\widerstand{R_D} = \mfrac{\spannung{U_D}}{\strom{I_D}} [\si{\ohm}]$ \\ \hline

                           \quad\quad\quad & \quad\quad\quad & \quad\quad\quad & \quad\quad\quad\\\hline
                            \quad\quad\quad  & \quad\quad\quad  & \quad\quad\quad &  \quad\quad\quad \\\hline
                            &   &   &  \\\hline
                             & \quad\quad\quad  &\quad\quad\quad  & \quad\quad\quad \\\hline
                             &  & \quad\quad\quad  & \quad\quad\quad \\\hline
                             &  & \quad\quad\quad  & \quad\quad\quad \\\hline
                            &  & \quad\quad\quad  & \quad\quad\quad \\\hline
                            &  & \quad\quad\quad  & \quad\quad\quad \\\hline
                             &  & \quad\quad\quad  & \quad\quad\quad \\\hline
                             &  & \quad\quad\quad  & \quad\quad\quad \\\hline
                            &  & \quad\quad\quad  & \quad\quad\quad \\\hline
                            &  & \quad\quad\quad  & \quad\quad\quad \\
                        \end{tabular}
                    \end{center}
                    \end{table}
                \end{center}
         \subsection{Auswertung}
             \subsubsection{Aufgabe 1} Stellen Sie die Kennlinie graphisch als Funktion $\strom{I} = f(\spannung{U_D})$ dar

             \subsubsection{Aufgabe 2} Nähern Sie die Kennlinie durch eine lineare Gerade an. Beschreiben Sie diesen Zusammenhang mathematisch als Funktion $\strom{I_D} = a + m \cdot \spannung{U_D}$. Was bedeutet die Konstante $a$ und $m$ ? An welchem Punkt schneiden sich Näherungsgerade und X-Achse, was beschreibt dieser Punkt?
             
        \section{Kennlinie einer Z-Diode (Z - Diode in Sperrrichtung)}
          \paragraph{Messaufbau:}
            \begin{itemize*}
                \item 1 Widerstand $\widerstand{R} = \SI{470}{\ohm}, \SI{2}{\watt}$
                \item 1 Z-Diode, Typ ZPD6V2 oder ZPD5V2
            \end{itemize*}
            \begin{center}
                \begin{circuitikz}[scale=1.3]
                    \draw
                    (0,2) to[R, l=$\widerstand{R}$] (1,2)
                    to[ammeter, i=$\strom{I}$] (3,2)
                    to[short] (5,2)
                    (5,0) to[zDo, v=$\spannung{U_D}$] (5,2)
                    (5,0)to[short] (0,0)
                    (3,2) to[myvoltmeter, l=$\spannung{U_V}$, *-*] (3,0)
                    ;
                \end{circuitikz}
            \end{center}

          \subsection{Messaufgaben}
            \subsubsection{Messaufgabe M1}
              \paragraph{Aufgabe:} Nehmen Sie die Kennlinie der Z - Diode $\spannung{U_D} = f(\strom{I_D})$ in Sperrrichtung auf
              \paragraph{Durchführung:}  Messschaltung aufbauen. Die Versorgungsspannung $\spannung{U_V} = \SI{0}{\volt}$ einstellen. Mit $\spannung{U_V}$ die Diodenspannung $\spannung{U_D}$ schrittweise erhöhen ( im Sperrbereich \SI{0,5}{\volt} Schritte) bis der Diodenstrom $\strom{I_D} = \SI{40}{\milli\ampere}$ beträgt. Messwerte für $\spannung{U_D}$ , $\strom{I_D}$ und $\widerstand{R_D} = \mfrac{\spannung{U_D}}{\strom{I_D}}$ protokollieren.
              \paragraph{Ergebnisse:}
                  \begin{center}
                      \begin{table}[!hbtp]
                          \caption{Messwertetabelle zur Messaufgabe 3.3.M1}
                          \label{tbl:messergebnisse3.3}
                          \renewcommand{\arraystretch}{1.3}
                          \begin{center}

                              \begin{tabular}{c|c|c|c}
                                  $\spannung{U_V} [\si{\volt}]$ &
                                  $\spannung{U_D}[\si{\volt}]$ &
                                  $\strom{I_D} [\si{\milli\ampere}]$ &
                                  $\widerstand{R_D} = \mfrac{\spannung{U_D}}{\strom{I_D}} [\si{\ohm}]$ \\ \hline

                                   &  &  & \\\hline
                                   &  &  & \\\hline
                                   &  &  &\\\hline
                                   &  &  & \\\hline
                                   &  & & \\\hline
                                   &  &  & \\\hline
                                   &  &  & \\
                              \end{tabular}
                          \end{center}
                      \end{table}
                  \end{center}

         \subsection{Auswertung}
            \subsubsection{Aufgabe 1:} Tragen Sie diesen Kennlinienteil in die graphische Darstellung aus Übung 3.2 ein. Beachten Sie dabei die Polarität der Diodenspannung und des Diodenstromes.

           \subsubsection{Aufgabe 2:} Nähern Sie diesen Kennlinienverlauf durch eine lineare Gerade an. Beschreiben Sie diesen Zusammenhang mathematisch als Funktion $\strom{I_D} = a + m \cdot \spannung{U_D}$. Was beschreiben die Konstanten $a$ und $m$ ? An welchem Punkt schneiden sich Näherungsgerade und X-Achse?
    

           \subsubsection{Aufgabe 3:} Interpretieren Sie die gesamte Kennlinie.
           


    \chapter{Zusammenschaltung linearer und nichtlinearer Netzwerke}


        \section{Belastung eines Festspannungsteilers mit einer Z - Diode}
          \paragraph{Messaufbau:}
            \begin{itemize*}
              \item 1 Widerstand $\widerstand{R_1} = \SI{470}{\ohm}, \SI{2}{\watt}$
              \item 1 Widerstand $\widerstand{R_2} = \SI{1}{\kilo\ohm}$
              \item 1 Widerstand $\widerstand{R_A} = \SI{100}{\ohm}$
              \item 1 Widerstand $\widerstand{R_A} = \SI{470}{\ohm}$
              \item 1 Widerstand $\widerstand{R_A} = \SI{1}{\kilo\ohm}$
              \item 1 Widerstand $\widerstand{R_A} = \SI{2}{\kilo\ohm}$
              \item 1 Z-Diode, Typ ZPD6V2 oder ZPD5V2
            \end{itemize*}
            \begin{center}
              \begin{circuitikz}[scale=1.3]
                  \draw
                     (0,2) to[R, l=$\widerstand{R_1}$, i=$\strom{I_E}$] (2,2)
                           to[R, l=$\widerstand{R_2}$] (2,0)
                           to[short] (0,0)
                     (0,2) to[open, v=$\spannung{U_V}$] (0,0)
                  ;
                  \draw [dash pattern=on 4pt off 4pt] (2,2)--(4,2);
                  \draw [dash pattern=on 4pt off 4pt] (4,0)--(2,0);
                  \draw [dash pattern=on 2pt off 2pt] (4,2)--(6,2);
                  \draw [dash pattern=on 2pt off 2pt] (6,0)--(4,0);
                  \draw
                      (4,0) to[zDo, v=$\spannung{U_2}$] (4,2)
                      (6,2) to[R, l=$\widerstand{R_A}$] (6,0)
                      (4,2) to[open, i=$\strom{I_A}$] (6,2)


                  ;
              \end{circuitikz}
            \end{center}

          \subsection{Messaufgaben}
              \subsubsection{Messaufgabe M1}
                  \paragraph{Aufgabe:} Kennlinie $\spannung{U_2}= f(\strom{I_A})$ des Spannungsteilers $\widerstand{R_1}$ ,$\widerstand{R_2}$ messtechnisch nach dem Verfahren der Ersatzspannungsquelle ermitteln.
                  \paragraph{Durchführung:} Messschaltung zunächst nur mit $\widerstand{R_1}$ und $\widerstand{R_2}$ aufbauen. Versorgungsspannung
                  $\spannung{U_V} = \SI{12}{\volt}$ einstellen. Ausgangskennlinie $\spannung{U_2}= f(\strom{I_A})$ ermitteln und notieren.

                  \paragraph{Ergebnisse:}
                  \begin{align*}
                      \spannung{U_2} &= \\
                      \strom{I_A} &= 
                  \end{align*}
        
              \subsubsection{Messaufgabe M2}
                  \paragraph{Aufgabe:}Spannungsteiler mit Z-Diode belasten. Spannung $\spannung{U_2}$, Strom $\strom{I_E}$ und $\strom{I_A}$ messen und aufschreiben.
                  \paragraph{Ergebnisse:}
                  \begin{align*}
                      \spannung{U_2} &= \\
                          \strom{I_E} &= \\
                          \strom{I_A} &= 
                  \end{align*}

            \subsubsection{Messaufgabe M3}
              \paragraph{Aufgabe:} Strom- Spannungs- Kennlinie $\spannung{U_2}= f(\strom{I_A})$ der Gesamtschaltung (Spannungsteiler mit Z-Diode) aufnehmen. Dazu Arbeitspunkte für $\widerstand{R_A} = \SI{100}{\ohm}, \SI{470}{\ohm}, \SI{1}{\kilo\ohm}$ und $\SI{2}{\kilo\ohm}$ messen.
              \paragraph{Ergebnisse:}
              \begin{center}
                  \begin{table}[!hbtp]
                      \caption{Messwertetabelle zur Messaufgabe 4.1.M3}
                      \label{tbl:messergebnisse4.1}
                      \renewcommand{\arraystretch}{1.3}
                      \begin{center}
                          \begin{tabular}{c|c|c|c}
                              $\widerstand{R_A} [\si{\ohm}]$&
                              $\strom{I_A} [\si{\milli\ampere}]$&
                              $\strom{I_E} [\si{\milli\ampere}]$&
                              $\spannung{U_2} [\si{\volt}]$\\ \hline

                              100 & \qquad\qquad\qquad & \qquad\qquad\qquad & \qquad\qquad\qquad\\\hline
                              470 &  & & \\\hline
                              1000 &  &  &\\\hline
                              2000 &  &  & \\
                          \end{tabular}
                      \end{center}
                  \end{table}
                  \end{center}

          \subsection{Auswertung}
            \subsubsection{Aufgabe 1:} Kennlinie $\spannung{U_2} = f(\strom{I_A})$ aus 4.1.1.M1 graphisch darstellen.

			\subsubsection{Aufgabe 2:} Kennlinie der Z-Diode aus Übung 3.3 in obige Kennlinie einzeichnen. Arbeitspunkt graphisch ermitteln. (Durchlass / Sperrrichtung angeben)
            
             
			\subsubsection{Aufgabe 3:} Zeichnen Sie die Kennlinie zur Messaufgabe 3.


			\subsubsection{Aufgabe 4:} Was bewirkt die Z - Diode?
            

			\subsubsection{Aufgabe 5:} Die obige Schaltung soll zur Spannungskonstanthaltung eines Verbrauchers $\widerstand{R_L} = \SI{470}{\ohm}$, bei der die Ausgangsspannung innerhalb des Bereiches $6,2 +/- 10\%$ liegen darf, eingesetzt werden. Zeichnen Sie dafür die Schaltung, berechnen Sie den Vorwiderstand $\widerstand{R_V}$ (Last bleibt konstant) und geben Sie den zulässigen Lastbereich an. (graphische, farbige Lösung).
            
        

    \listoftables
\end{document}
